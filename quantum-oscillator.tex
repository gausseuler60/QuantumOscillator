\documentclass[%
aps, pra,
% jmp,
% bmf,
% sd,
% rsi,
amsmath,amssymb,
%preprint,%
reprint,%
%author-year,%
%author-numerical,%
% Conference Proceedings
superscriptaddress
]{revtex4-2}

\usepackage{amsmath}
\usepackage{graphicx}
\usepackage{subcaption}
\graphicspath{ {./images/} }


\begin{document}

%opening
\title{Comparing dynamics of dissipative nonlinear coupled oscillators in classical and quantum mechanics}
\author{I.?. Nazestkin}
	\affiliation{ 
	Moscow Institute of Physics and Technology, Dolgoprundiy, Russia
}
\affiliation{ 
	Russian Quantum Center, National University of Science and Technology MISIS, 119049 Moscow, Russia
}%
\author{Y.?. Babich}
	\affiliation{ 
	Moscow Institute of Physics and Technology, Dolgoprundiy, Russia
}
\author{G.P. Fedorov}
	\affiliation{ 
	Moscow Institute of Physics and Technology, Dolgoprundiy, Russia
}
\affiliation{ 
	Russian Quantum Center, National University of Science and Technology MISIS, 119049 Moscow, Russia
}%

\begin{abstract}
We discuss similarities and differences in the behavior of models describing two coupled nonlinear oscillators in the classical and quantum description. Starting from the linear case when the dynamics is similar for both limits, we obtain analytical solutions for a useful case when friction in one of the oscillators is dominating. We show that a concise solution is attained for this case within RWA and that the vacuum Rabi-splitting is reproduced by classical mechanics. Next, we turn on the cosine nonlinearity in the system and compare the transmission spectra for high-amplitude driving. We highlight the similarities between quantum Rabi oscillations and the amplitude beatings in the classical nonlinear system. Finally, we discuss the differences between the preditions of the classical and quantum physics on the cross-Kerr interaction between the oscillators when they are detuned. We believe that this work will be useful as an introductory guide for students entering the field of superconducting quantum computing.
\end{abstract}

\maketitle

\section{Introduction}

\section{Mechanical oscillator, simple Hamiltonian system}

As a mechanical case, let's consider a mechanical oscillator with two interacting objects of mass $M$ connected with spring of elasticity $k_g$. A resistance force $F = -\beta {\dot x}^2 $ acts on the second object.
\begin{figure}[h]
	\centering
	 %\includegraphics{Fig_2.png}
	 \caption{Mechanical oscillator setup}
\end{figure}\newline

We will use generalized coordinates $x_1(t)$ and $x_2(t)$ as displacements from balance position for  object 1 and 2, respctively.\newline
First, consider a system without dissipations: 
System Lagrangian (without dissipation):\newline
\begin{align*}
L &= \frac{M}{2}\left({\dot x_1}^2 + {\dot x_2}^2\right) - \frac{k}{2}\left(x_1^2 + x_2^2\right) - \frac{k_g}{2}\left(x_2-x_1\right)^2\\
\end{align*}
System Hamiltonian:
\begin{equation}
H = \frac{p_1^2}{2M} + \frac{p_2^2}{2M} + \frac{k}{2}\left(x_1^2 + x_2^2\right) + \frac{k_g}{2}\left(x_2 - x_1\right)^2
\end{equation}
Following motion equations:\newline
\begin{align*}
	\begin{cases}
		\dot x_1 = \frac{p_1}{M}
		\\
		\dot x_2 = \frac{p_2}{M}
		\\
		\dot p_1 = -(k+k_g)x_1 + k_g x_2
		\\
		\dot p_2 = k_gx_1 - x_2(k+k_g)
	\end{cases}
\end{align*}
Now add the dissipative force to the second equation:
\begin{align*}
	\begin{cases}
		\dot x_1 = \frac{p_1}{M}
		\\
		\dot x_2 = \frac{p_2}{M}
		\\
		\dot p_1 = -(k+k_g)x_1 + k_g x_2
		\\
		\dot p_2 = k_gx_1 - x_2(k+k_g) - \beta \dot x_1
	\end{cases}
\end{align*}
To investigate system damping behavior depending on different parameter values, plot real and complex parts of system eigenvalues:
\begin{figure}[h]
	\centering
	\includegraphics[scale=0.35]{Fig_2.pdf}
	\caption{System damping depending on resistance coefficient $\beta$}
\end{figure}\newline

As can be easily noticed, there are two components which will never be equal to zero for any arbitrary large $\beta$. This is a real part ($Re \lambda_1$) and imaginary one ($Im \lambda_4$). Thus, motion equation will always contain a damping exponent (from real part) and sine/cosine oscillating part. With $\beta$ increment a quotient in an exponent will increase, leading to more intensive damping.

Expressions for $x_1(t)$ and $x_2(t)$ are very complicated to be written down. Solution was found numerically with Adams method using Python (\textit{scipy.integrate.odeint} method from SciPy library). Results are shown in fig.3.
\begin{figure*}
	\includegraphics[scale=0.3]{Fig_3_a.pdf}
	
	\includegraphics[scale=0.3]{Fig_3_b.pdf}
	\caption{Motion differential equations solutions, $x(t)$. (a) $k \sim k_g$, different resistance ($\beta$). (b) (left) $k<<k_g$ (strong interaction), $\beta=0$. Two objects are oscillating together like one object. (center) $k>>k_g$ (weak interaction), $\beta=0$. Beats are observed. (right) $k>>k_g, \beta \ne 0$, damping beats.}
\end{figure*}

\section{Rotating wave approximation}

To simplify equations system and write down analytical solution, an attempt to use a method frequently used in quantum mechanics was made. We used new coordinates which will be constant if it is no interaction ($k_g=0$).
\newline We introduced new complex conjugated coordinates $a$ and $a^\dag$. (In quantum mechanics, they are particle creation and annihilation operators). They can be expressed from $x$ and $p$ using following equations:
\begin{align*}
	\begin{cases}
		\hat{x} = \sqrt{\frac{\hbar}{2M\omega}}\left(a+a^\dag\right)
		\\
		\hat{p} = i\sqrt{\frac{\hbar M \omega}{2}}\left(a-a^\dag\right)
	\end{cases}	
\end{align*}

In our mechanical system, $\hat{x}$ and $\hat{p}$  operators are just scalars, $x(t)$ and $p(t)$. Consider $\hbar=1$.\newline

Check whether this transform is canonical and find its valence.
For canonicity, the following conditions must be met:
\begin{align*}
	\begin{cases}
		\{p_1(a_1, a^\dag_1), p_2(a_2, a^\dag_2)\}=0
		\\
		\{x_1(a_1, a^\dag_1), x_2(a_2, a^\dag_2)\}=0
		\\
		\{x_1(a_1, a^\dag_1), p_1(a_1, a^\dag_1)\}=c
		\\
		\{x_2(a_2, a^\dag_2), p_2(a_2, a^\dag_2)\}=c
		\\
		\{x_1(a_1, a^\dag_1), p_2(a_2, a^\dag_2)\}=0
		\\
		\{x_2(a_2, a^\dag_2), p_1(a_1, a^\dag_1)\}=0
	\end{cases}
\end{align*}
These conditions can be easily checked:

$\{a_1(x_1, p_1), a_1^\dag(x_1, p_1)\}=\left(\frac{\partial a_1}{\partial x_1} \frac{\partial a_1^\dag}{\partial p_1} - \frac{\partial a_1}{\partial p_1} \frac{\partial a_1^\dag}{\partial x_1}\right) + \left(\frac{\partial a_1}{\partial x_2} \frac{\partial a_1^\dag}{\partial p_2} - \frac{\partial a_1}{\partial p_2} \frac{\partial a_1^\dag}{\partial x_2} \right) = \sqrt{\frac{M\omega}{2}} \frac{1}{i\sqrt{2M\omega}}+\frac{1}{i \sqrt{2M \omega}} \sqrt{\frac{M \omega}{2}}=-i$\newline

These are canonical equations with valence $i$.
So, Hamiltonian (1) will look like:
\begin{align*}
	H=-i a_1 a_1^\dag \omega - i a_2 a_2^\dag \omega + ig(a_1+a_1^\dag)(a_2+a_2^\dag),
\end{align*}
where $g = \frac{k_g}{2\sqrt{M}\sqrt{k+k_g}}$ - interaction constant.\newline

Now write down Hamilton equations. We will write only equations for $a_1$ and $a_2$, because another two equations are just complex conjugates of first ones and are not necessary.
\begin{align*}
	\begin{cases}
		\dot a_1 = -a_1 i \omega + ig(a_2 + a_2^\dag)
		\\
		\dot a_2 = -a_2 i \omega + ig(a_1 + a_1^\dag)
	\end{cases}
\end{align*}
Now make the following ansatz: $a_1(t) = a_1(t)e^{-i\omega t}, a_2(t) = a_2(t)e^{-i\omega t}$:
\begin{align*}
	\begin{cases}
		\dot a_1(t) = ig(a_2(t)e^{-i\omega t} + a_2(t)e^{i\omega t})
		\\
		\dot a_2(t) = ig(a_1(t)e^{-i\omega t} + a_1(t)e^{i\omega t})
	\end{cases}
\end{align*}

In case of weak coupling ($k>>k_g$) one can apply rotating wave approximation (RWA). It allows to remove slowly oscillating terms from equations. Here, $e^{i\omega t} \approx 0$, so then:
\begin{align*}
	\begin{cases}
		\dot a_1 = iga_2
		\\
		\dot a_2 = iga_1
	\end{cases}
\end{align*}
And with dissipation:
\begin{align*}
	\begin{cases}
		\dot a_1 = iga_2
		\\
		\dot a_2 = iga_1 - \frac{\beta}{2}a_2 
	\end{cases}
\end{align*}
System eigenvalues:
\begin{align*}
	\lambda_{1,2} = -\frac{\beta}{4} \pm \frac{\sqrt{\beta^2-16g^2}}{4}
\end{align*}
Real and complex parts of eigenvalues depending on resistance coefficient $\beta$ (fig. 4):
\begin{figure}[h]
	\centering
	\includegraphics[scale=0.35]{Fig_4.pdf}
	\caption{System damping depending on resistance coefficient $\beta$}
\end{figure}\newline
Corresponding eigenvectors:
\begin{align*}
		v_{1,2} = \left(-\frac{4ig}{b \mp \sqrt{b^2-16g^2}}, 1	\right)
\end{align*}
Now a solution can be written down in a more simple way. Solutions are shown on fig. 5. As expected, RWA can yield exact result only if $k>>k_g$
\begin{figure}[h]
	\centering
	\includegraphics[scale=0.3]{Fig_5.pdf}
	\caption{System motion differential equations solved classically and using RWA. (a): {$k>>k_g$}, approximation yields good quality. (b): {$k \sim k_g$}, significant error increases over time.}
\end{figure}

\end{document}

